\documentclass[25pt, portrait]{tikzposter}
\usepackage{authblk}
\usepackage{tikz}
\usepackage{adjustbox}
\usepackage{textcomp}
\usepackage{multicol}
\usepackage{amsmath}
% \usepackage{amssymb}
\usetikzlibrary{shapes, arrows.meta}
\usetheme{Default}
% \definecolorpalette{Custom}{
%     % GrayBlueYellow
%     \definecolor{colorOne}{HTML}{DDDDDD}
%     \definecolor{colorTwo}{HTML}{153DA6}
%     \definecolor{colorThree}{HTML}{FFDC00}  
% }
\usecolorpalette{Custom}
  

\title{\parbox{.85\textwidth}{ \begin{center}Multi-dimensional Femtosecond-laser induced dynamics of CO on metals:\end{center}
\begin{center}\LARGE accounting for electronic friction and surface motion with combined models
 \end{center}}}
\author[1]{\underline{Robert Scholz}}
\author[1]{Peter Saalfrank}
\author[2]{Ivor Lon\v{c}ari\'c}
\author[3]{Jean Cristophe Tremblay}
\author[3]{Gernot F\"uchsel}
\author[1]{Gereon Flo\ss{}}

\affil[1]{Institut f\"ur Chemie, Universit\"at Potsdam, Karl-Liebknecht-Str. 24-25, 14476 Potsdam, Germany}
\affil[2]{Ru\dj{}er Bo\v{s}kovi\'{c} Institute, Div. of Theor. Physics, Bijeni\v{c}ka cesta 54, 10000 Zagreb, Croatia}
\affil[3]{Freie Universit\"at, Inst. f\"ur Chemie und Biochemie, Takustr. 3, 14195 Berlin, Germany}

% \affil[4]{Centro de F\'{\i}sica de Materiales CFM/MPC (CSIC-UPV/EHU), Paseo Manuel de Lardizabal 5, 20018 Donostia-San Sebasti\'an, Spain}
% \affil[5]{Departamento de F\'{\i}sica de Materiales,
% Facultad de Qu\'{\i}micas, 
% Universidad del Pa\'{i}s Vasco (UPV/EHU), Apartado 1072, 20080 San Sebasti\'an, Spain}
% \affil[6]{Donostia International Physics Center DIPC, P. Manuel de Lardizabal 4, 20018 San Sebasti\'an, Spain}

\makeatletter
\def\maketitle{\AB@maketitle}
\makeatother
\renewcommand{\Authfont}{\Large}
\renewcommand{\Affilfont}{\normalsize}

% \titlegraphic{\includegraphics{Header.png}}
\settitle{ \centering \vbox{
\@titlegraphic \\[\TP@titlegraphictotitledistance] \centering
\color{titlefgcolor} {\bfseries \huge \sc \@title \par}
\vspace*{1em}
{ \@author \par}
}}


\begin{document}
  \maketitle[width=0.9\textwidth]
%   \includegraphics{Header.png}
  \begin{columns}
    \column{0.38}
      \block{Introduction}{
		\innerblock{Motivation}{
		  \begin{minipage}[t]{.62\colwidth}
			\begin{itemize}
			  \item {\bf Aim:} gain {\bf precise understanding} of \newline\hspace*{3cm}{\bf adsorbate bonding} on {\bf metals}
                            \newline\hspace*{2.7 cm} $\Rightarrow$ Important for {\bf Catalysis}
			  \item {\bf Why femtosecond(fs)-lasers?} 
			  \begin{itemize}
				\item {\bf produce} non-equilibrium {\bf 2-T-states} \newline $\Rightarrow$ {\bf different than} normal {\bf heating}
				\item further {\bf tool} besides STM and scattering 
				\item {\bf direct} future {\bf applications} possible \newline$\Rightarrow$ {\bf ``femtochemistry''}\cite{bonn:99} 
			  \end{itemize}
			\end{itemize}  
		  \end{minipage}
		  \begin{adjustbox}{valign=t}
			\begin{minipage}[t]{.27\colwidth}
			  \begin{flushright}
% 				\includegraphics[width=.18\colwidth]{abb/nanofemto.png}
				%$\,$\\[.6\baselineskip]
				\includegraphics[width=.24\colwidth]{abb/CO_CO2.png}
				\begin{small}Femtochemistry example                                                \end{small}
			  \end{flushright}
			\end{minipage}
		  \end{adjustbox}
		  \vspace*{-2.3\baselineskip}
                        \begin{itemize}
                          \item {\bf Why CO/Ru(001)} and {\bf CO/Cu(100)}? 
                          \begin{itemize}
                            \item both are well studied {\bf model systems} 
                            \item recently, interesting {\bf fs-laser experiments}\cite{nilsson:13}\cite{inoue:16} 
                            \item also, {\bf ab-initio} based {\bf 6-dim. potentials} available \cite{fuechsel:14}\cite{marqu:10}
                          \end{itemize}                   
                        \end{itemize}
% 		  \begin{itemize}
% 			\item specific motivation for system CO/Ru(0001)
% 			\begin{itemize}
% 			  \item experimentally well studied regarding fs-laser irradiation, e.g. \cite{funk:00, nilsson:13}
% 			  \item fulldimensional \textit{ab-initio} potential recently developed in our group\cite{fuechsel:14}
% 			  \item details of this indicate interpretation of experiment \cite{nilsson:13} may be wrong
% 			\end{itemize}
% 		  \end{itemize}		  		    
		}
		\vspace{.5cm}

		\innerblock{How do fs-lasers affect adsorbate-metal systems?}{
% 		  \innerblock{How fs-laser-irradiation affects metal surfaces}{
% 		  \innerblock{What happens upon FL-irradiation of metal surfaces?}{
		  \includegraphics[width=.90\colwidth]{abb/modifiedSurfScheme.png}		    
		  \begin{minipage}[t]{.60\colwidth}
			\begin{itemize}
% 			  \item metals: 
% 			  %can be described as 
% % 			    consist of an
% 					ion lattice plus quasi-free electron gas 
			  \item only {\bf electrons} of metal {\bf absorb laser}
			  \item {\bf electron-hole pairs} thermalize fast
					\newline$\Rightarrow$ {\bf ``hot''} Fermi-Dirac-distribution
% 				  \item Fermi-Dirac schon hier?
						% \vbox syntax? (for the graphic)
			  \item electrons transfer energy to ion lattice, via \raisebox{-.18\baselineskip}{\includegraphics[height=.8\baselineskip]{abb/one.png}}
			  \hspace{-0.4cm} \textbf{electron-phonon coupling}%\newline (phonons = lattice vibrations; quasi-particles)
                          \item {\bf equilibration within  ps-timescale}			
			  \item[$\Rightarrow$] Thus, for few ps {\bf two temperatures}:
			  \begin{itemize}
				\item $T_\mathrm{el}$ - electron temperature
				\item $T_\mathrm{ph}$ - phonon temperature
% 					\item Two-Temperature Model (TTM)\cite{anisimov:74}
			  \end{itemize}
			  \item both can {\bf couple} to adsorbed {\bf molecule}
			  \item low electron heat {\bf capacity} $\Rightarrow$ $T_\mathrm{el}$ higher
			\end{itemize}
% 			\vspace*{-0.6\baselineskip}
% 			\begin{flushright}
%                           \begin{small}
%                             \hspace*{1cm} Two-Temperature Model (2TM) \cite{anisimov:74} 
%                           \end{small}
%                         \end{flushright}
		  \end{minipage}
		  \begin{adjustbox}{valign=t}
			\begin{minipage}[t]{.28\colwidth}
			  \begin{flushright}
% 				\includegraphics[width=.18\colwidth]{abb/elgas.png}$\,$\\[.2\baselineskip]
				\includegraphics[width=.27\colwidth]{abb/fermi.png}\vspace*{-.3\baselineskip}\\
				\begin{small}Fermi-Dirac-Distribution                                                   \end{small}\vspace*{.5\baselineskip}\\
				\includegraphics[width=.29\colwidth]{abb/TTM_1pulse.eps}
			  \end{flushright}
			\end{minipage}
		  \end{adjustbox}
%  		  \begin{itemize}
%                     \item can be simulated by {\bf Two-Temperature Model} \cite{anisimov:74} 
%  		  % % 			\item can be simulated using a Two-Temperature Model (2TM)%(TTM)
% % % 				  \cite{anisimov:74} (see right)
% % 				  %consisting of two differential Equations 
%                	  \end{itemize}
		}
% 		  \includegraphics[width=.4\colwidth]{abb/TTM_1pulse.eps}
	  }
	\column{0.62}
	  \block{Models and Methods}{
		\innerblock{Basis of the Dynamics: the Six-dimensional Potential Energy Surfaces (PES)\cite{fuechsel:14}\cite{marqu:10}}{
% 		  \begin{minipage}[t]{.52\colwidth}
			\begin{adjustbox}{valign=t}
			  \begin{minipage}[t]{.18\colwidth}
% 				  \vspace*{.5\baselineskip}
				\includegraphics[width=.18\colwidth]{abb/6dimScheme.png}\vspace*{-0.15\baselineskip}\\
				\hspace*{.02\colwidth}\begin{small} 
                                  The six dimensions of CO
                                \end{small}
			  \end{minipage}

			\end{adjustbox}
			\begin{minipage}[t]{.33\colwidth}
			  \begin{itemize}
				\item {\bf precomputed} with DFT (GGA)
				\item all {\bf six dimensions} of the adsorbate 
				\item {\bf analytical} $\Rightarrow$ {\bf very fast} % (once constructed)
				\newline $\Rightarrow$ {\bf many trajectories} possible 
				\item but: surface frozen $\Rightarrow$ {\bf no phonons} 
			  \end{itemize}
			\end{minipage}
% 		  \end{minipage}
		  \begin{adjustbox}{valign=t}
			\begin{minipage}[t]{.21\colwidth}
			  \vspace*{-0.2\baselineskip}\includegraphics[width=.195\colwidth]{abb/fig_1b.pdf}\vspace*{-0.615\baselineskip}\\
			  \hspace*{.025\colwidth}\begin{small}Potential for CO/Ru(001)                                          \end{small}
			\end{minipage}
		  \end{adjustbox}
		  \begin{adjustbox}{valign=t}
                        \begin{minipage}[t]{.18\colwidth}
                         \includegraphics[width=.185\colwidth]{abb/grid640_pot_3.png}\vspace*{-0.46\baselineskip}\\
                          \hspace*{1cm}\begin{small}Potential for CO/Cu(100)                                          \end{small}
                        \end{minipage}
                  \end{adjustbox}
		}
		\vspace{.5cm}
	  
		\innerblock{Two-Temperature Model (2TM)\cite{anisimov:74}}{
		  \begin{minipage}[t]{.68\colwidth}
			\begin{itemize}
			  \item describes {\bf interaction} of {\bf electrons} with {\bf phonons} and {\bf laser} 
			\end{itemize}			  
			\begin{center}
			  \includegraphics[width=.65\colwidth]{abb/2TM_equs.png}
			\end{center}
		  \end{minipage}
		  \begin{adjustbox}{valign=t}
			\begin{minipage}[t]{.25\colwidth}
			  \includegraphics[width=.25\colwidth]{abb/Part1SurfScheme.png}
% 				\includegraphics[width=.3\colwidth]{abb/2TM_equs.png}
			\end{minipage}
		  \end{adjustbox}
% 			\end{adjustbox}
% 		  \vspace{-1.2\baselineskip}
		  \begin{itemize}
			\item[$\Rightarrow$] {\bf get $\boldsymbol{T}_\mathbf{el}$ and $\boldsymbol{T}_\mathbf{ph}$ as $\boldsymbol{f(z,t)}$} from laser parameters and material properties 
% 			{\normalsize
% 			\begin{itemize}
% 			  \begin{minipage}[t]{.52\colwidth}
% 				\item laser wavelength $\lambda$ (affects penetretion depth into material)
% 				\item (effective) absorbed fluence $F$ (energy/area)
% 				\item pulse duration $\tau$ (all three appear in the ``source term'' $S(z, t)$)
% 			  \end{minipage}
% 			  \begin{minipage}[t]{.5\colwidth}
% 				\item electron and phonon heat capacities $C_\mathrm{el}$ and $C_\mathrm{ph}$
% 				\item electron heat conductivity $\kappa$ 			
% 				\item electron-phonon coupling constant $g$
% 			  \end{minipage}
% 			\end{itemize}				
% 			}
		  \end{itemize}	

		}
		\vspace{.5cm}
		  
		\innerblock{Electronic Friction: Langevin Dynamics\cite{headgordon:95} and Local~Density~Friction~Approximation~(LDFA)\cite{juaristi:08}}{
		  \begin{adjustbox}{valign=t}
			\begin{minipage}[t]{.26\colwidth}
			  \includegraphics[width=.25\colwidth]{abb/Part2SurfScheme.png}
% 				\includegraphics[width=.3\colwidth]{abb/2TM_equs.png}
			\end{minipage}
		  \end{adjustbox}
		  \begin{minipage}[t]{.6\colwidth}
			\begin{itemize}
			  \item Langevin equation of motion, a stochastical differential equation:%\\[-.9\baselineskip]
% 				\vspace{.2\baselineskip}
% 				\item describes interaction of  electron-hole-pairs with molecule
			\end{itemize}
			\vspace*{-.1\baselineskip}			  
			\includegraphics[width=.65\colwidth]{abb/Langevin_eq.png}
		  \end{minipage}
		  \vspace*{-.3\baselineskip}
		  \begin{itemize}
			\item describes {\bf movement of CO} and {\bf interaction with electron-hole pairs} (friction and excitation)
			\item {\bf Local Density Friction Approx.} (LDFA): simple {\bf model} to get {\bf friction coefficients} $\boldsymbol{\eta}_{\mathbf{el},\boldsymbol{k}}$
			\begin{itemize}
% 			    \item ``Local Density'' $\Rightarrow$ homogenous free electron gas (FEG)
			  \item {\bf Atom $\boldsymbol{k}$ embedded in free electron gas} with density of bare surface at current position $\underline{r}_k$
			\end{itemize}
			  \item {\bf Random forces} $\underline{\boldsymbol{R}}_{\mathbf{el},\boldsymbol{k}}$: white noise, {\bf dependent on} both $\boldsymbol{\eta}_{\mathbf{el},\boldsymbol{k}}$ (from LDFA) and $\boldsymbol{T}_\mathbf{el}$ (from 2TM)
		  \end{itemize}

% 			\begin{adjustbox}{valign=t}
% 			  \begin{minipage}[t]{.5\colwidth}
% % 				\includegraphics[width=.15\colwidth]{abb/Part1SurfScheme.png}\\[.3\baselineskip]
% 				\includegraphics[width=.5\colwidth]{abb/Langevin_eq.png}
% 		      \end{minipage}
% 			\end{adjustbox}
		}
		\vspace{.5cm}
		
		\innerblock{Inclusion of Phonons: Generalized Langevin Oscillator(GLO)-model\cite{busnengo:05}}{
		  \begin{adjustbox}{valign=t}
			\begin{minipage}[t]{.24\colwidth}
			  \includegraphics[width=.25\colwidth]{abb/Part3SurfScheme.png}
% 				\includegraphics[width=.3\colwidth]{abb/2TM_equs.png}
			\end{minipage}
		  \end{adjustbox}
		  \begin{minipage}[t]{.7\colwidth}
			\begin{itemize}
			  \item influence of {\bf phonons} effectifely {\bf modeled} ({\bf augments frozen surface})
			  \item {\bf entire surface} understood as {\bf 3D oscillator} (coords. $\underline{r}_s$, mass 1 atom)
			  \item {\bf coupling} to molecule {\bf via shifting}: %of the PES:
			  $\; V_\mathrm{GLO}(\underline{r}_\mathrm{C},\underline{r}_\mathrm{O};\underline{r}_s)=V(\underline{r}_\mathrm{C}-\underline{r}_s,\underline{r}_\mathrm{O}-\underline{r}_s)$ 
			  \item additionally coupled to {\bf ghost oscillator} $\underline{r}_g$, {\bf models} influence of {\bf bulk} 
			  \begin{itemize}
				\item ghost oscillator is subject to friction $\eta_\mathrm{ph}$ and random forces $\underline{R}_\mathrm{ph}(T_\mathrm{ph})$
			  \end{itemize}
			\end{itemize}
% 			  {\Large
% 			  \begin{equation*}
% 			    V_\mathrm{GLO}(\underline{r}_\mathrm{C},\underline{r}_\mathrm{O};\underline{r}_s)=V(\underline{r}_\mathrm{C}-\underline{r}_s,\underline{r}_\mathrm{O}-\underline{r}_s) 
% 			  \end{equation*}
% 			  }
% 			  \vspace*{-.1\baselineskip}			  
% 			  \includegraphics[width=.65\colwidth]{abb/Langevin_eq.png}
		  \end{minipage}
		}
	  }	
  \end{columns}
%     \vspace*{1\baselineskip}
  \begin{columns}
  \column{.66}
  \block[titleoffsety=3cm,bodyoffsety=3cm,titleoffsetx=-.182\colwidth,titlewidthscale=.62,]{Results}{
	\begin{adjustbox}{valign=t}
	\begin{minipage}[t]{.49\colwidth}
	  \innerblock{Desorption (Data for Ru)}{
% 		\begin{itemize}
% 		  \item 
% 		\end{itemize}
		\begin{adjustbox}{valign=t}
		  \begin{minipage}[t]{.26\colwidth}
			\includegraphics[width=.26\colwidth]{abb/fig_8.png}
		  \end{minipage}
		\end{adjustbox} 
		\begin{minipage}[t]{.2\colwidth}
		  \begin{itemize}
		    \item desorption mainly during first 50 ps
		    \item fluence dependence of desorption yield close to experiment
% 		    \item no ``precursor state'' as suggested by \cite{nilsson:13}
		    \item no barrier in PMF
		  \end{itemize}
		\end{minipage}


		\includegraphics[width=.24\colwidth]{abb/fig_9.eps}
		\includegraphics[width=.19\colwidth]{abb/fig_11a.eps}		
	  }	  
	\end{minipage}
	\end{adjustbox}
	\hspace*{.005\colwidth}
	\begin{adjustbox}{valign=t}
	  \begin{minipage}[t]{.46\colwidth}
	  	\innerblock{Diffusion (Data for Ru, but Cu similar)}{
		  \begin{adjustbox}{valign=t}
			\begin{minipage}[t]{.26\colwidth}	  	
			  \includegraphics[width=.26\colwidth]{abb/Trajektorie.png}
			\end{minipage}
		  \end{adjustbox} 
		  \begin{minipage}[t]{.16\colwidth}
			\begin{itemize}
			  \item typical trajectory:
			   hops between top \\sites and vibration 
			  \item increase in $\theta$-angle
					when CO moves away from top
			  \item overall, very large diffusion 
			\end{itemize}
		  \end{minipage}

		  



		  \vspace{.5\baselineskip}
		  \begin{itemize}
		    \item also, nonisotropic diffusion behaviour observed:
		    \item dynamical trapping effect at hcp site predicted
% 		    \item[$\Rightarrow$] possible alternative explanation to ``precursor state'' \cite{nilsson:13}
		  \end{itemize}

		  \includegraphics[width=.445\colwidth]{abb/movie.png}
		}	    
	  \end{minipage}
	\end{adjustbox}	  
	\innerblock{Vibrations (Data for Cu)}{
          \begin{adjustbox}{valign=t}          
            \begin{minipage}[t]{.26\colwidth}~\\
              \begin{center}\begin{itemize}
                              \item Frequency-shift from \newline time-resolved SFG    
                            \end{itemize}
           \end{center}
            \end{minipage}
          \end{adjustbox}
          \begin{adjustbox}{valign=t}          
            \begin{minipage}[t]{.2\colwidth}
              \includegraphics[height=4\baselineskip]{abb/Inoue.png}
            \end{minipage}
          \end{adjustbox}          
          \begin{adjustbox}{valign=t}          
            \begin{minipage}[t]{.2\colwidth}
              \includegraphics[height=4\baselineskip]{abb/Fourier.pdf}  
            \end{minipage}
          \end{adjustbox}         
          \begin{adjustbox}{valign=t}          
            \begin{minipage}[t]{.26\colwidth}~\\
              \begin{center}\begin{itemize}
                              \item Preliminary results \newline from our dynamics  
                            \end{itemize}

             \end{center} 
            \end{minipage}
          \end{adjustbox}          
	}
  }
  \column{.33}
  \block{}{
	\innerblock{Conclusions}{
	\begin{itemize}
	  \item 6D Langevin dynamics of CO on Ru and Cu
	  \item based on first principles, no ``free'' parameters
	  \item accounting for (via LDFA) electronic friction, hot electron excitation and (via GLO) substrate motion
	  \item allows for detailed time- and space-resolved insights
	  \item no physisorbed state, molecules desorp directly 
	\end{itemize}
	}
	\vspace{1.7cm}
	\innerblock{Outlook}{
	\begin{itemize}
% 	  \item 
	  \item better electonic friction ($\eta(T_\mathrm{el})$ and beyond LDFA)
	  $\Rightarrow$ Long term goal: use tensorial friction (exact)
	  \item non-equilibrium lattice model (NLM) instead 2TM 
	  \item simulate other coveages (bigger super cells)
	  \item simulate bigger systems (CO + H; hydrocarbons) 
	  \item include interaction between adsorbate molecules
	\end{itemize}

	}
  }
  \end{columns}
      %   \bibliography{/perm/scholz/phd/lit/biblio}
  \block{}{
    \begin{multicols}{3}\footnotesize
      \begin{thebibliography}{X}\small
		\bibitem{bonn:99}
		M. Bonn, S. Funk, Ch. Hess, D.N. Denzler %,
		\emph{et al.}, %, C. Stampfl, M. Scheffler, M. Wolf, G. Ertl,  
		\emph{Science} \textbf{285}, 1042 (1999).

		
% 		\bibitem{funk:00}
% % 		S. Funk, M. Bonn, D. N. Denzler, C. Hess \emph{et al.}, 
% 		S. Funk, M. Bonn, D.N. Denzler, C. Hess \emph{et al.}, %M. Wolf and G. Ertl, 
% 		\emph{J.Chem.Phys.} \textbf{112}, 9888 (2000).
		
		\bibitem{nilsson:13}
		  M. Dell'Angela, T. Anniyev, M. Beye, R. Coffee \emph{et al.}, 
% 		  M. Dell'Angela, T. Anniyev, M. Beye, R. Coffee, A. F\"ohlisch \emph{et al.}, 
		  \emph{Science} \textbf{339}, 1302 (2013).
		
		\bibitem{inoue:16} 
                  K. Inoue, K. Watanabe, T. Sugimoto \emph{et al.},  \emph{Phys.Rev.Lett.} \textbf{117}, 186101 (2016).
		
		\bibitem{fuechsel:14}
		  G. F�chsel, J.C. Tremblay, and P. Saalfrank, 
		  \emph{J.Chem.Phys.} \textbf{141}, 094704 (2014).
		
		\bibitem{marqu:10}
                  R. Marquardt, F. Cuvelier, R.O. Olsen \emph{et al.}, \emph{J.Chem.Phys.} \textbf{132}, 074108 (2010).
		
		\bibitem{anisimov:74}
		  S.I. Anisimov, B.L. Kapeliovich and T.L. Perelman,
		  \emph{Sov.Phys.-JETP} \textbf{39}, 375 (1974).
		  
		\bibitem{headgordon:95}  
		M. Head-Gordon and J.C. Tully, \emph{J.Chem.Phys.} \textbf{103}, 10137 (1995).
		
		\bibitem{juaristi:08}
		J.I. Juaristi, M. Alducin, R. D�ez Mui�o \emph{et al.}, 
% 		J. I. Juaristi, M. Alducin, R. D�ez Mui�o, H. F. Busnengo and A. Salin, 
		\emph{Phys.Rev.Lett.} \textbf{100}, 116102 (2008).
		
% 		\bibitem{adelman:76}
% 		S.A. Adelman and J.D. Doll, \emph{J.Chem.Phys.} \textbf{64}, 2375 (1976).
		
% 		\bibitem{tully:80}
% 		J.C. Tully, \emph{J.Chem.Phys.} \textbf{73}, 1975 (1980).
		
		\bibitem{busnengo:05}
		H.F. Busnengo, M.A. Di C�sare, W. Dong \emph{et al.},
		\emph{Phys.Rev.B} \textbf{72}, 125411 (2005).
		

		
      \end{thebibliography}  
    \end{multicols}
  }
\end{document}


% \makeatletter
% \renewenvironment{tikzfigure}[1][]{
%   \def \rememberparameter{#1}
%   \vspace{10pt}
%   \refstepcounter{figurecounter}
%   \begin{center}
%   }{
%     \ifx\rememberparameter\@empty
%     \else %nothing
%     \\[10pt]
%     {\large Fig.~\thefigurecounter: \rememberparameter}
%     \fi
%   \end{center}
% }
% \makeatother
